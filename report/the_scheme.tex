%\chapter{The scheme}
\section{Delta Correlating Prediction Table}
The algorithm for DCPT (Delta Correlating Prediction Table) is based on the pseudo code from \cite{dcptpaper}.
The implementation is very near RPT, but the difference is that it looks for a pattern instead of one stride.
When the it gets a request, it saves the (PC) program counter, and the stride in a table connected to each PC. When it has more than two elements in the list with strides, it starts looking for patterns.
If it finds a pattern in the list, it saves all the adresses that possibly should be prefetched in another list with candidates for prefetch.
Then it runs through the list and checks if the adresses is not in the cache, in the mshr or that the adress is out of memory. If all the requirements are met, it issues a prefetch for it.

Attempts to tweak this implementation:
- Make a copy of the queue to keep track of which adresses that still aren't fetched, to avoid "double fetching".
- Make a lru which prevents the prefetcher from throwing out recently used history from the list.
- Adjustments to the size of the list, and size of the delta list


