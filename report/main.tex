\documentclass[12pt,journal,compsoc]{IEEEtran}
\providecommand{\PSforPDF}[1]{#1}
\usepackage{graphicx}
\usepackage{algorithm2e}
\usepackage{verbatim}
\newcommand\MYhyperrefoptions{bookmarks=true,bookmarksnumbered=true,
pdfpagemode={UseOutlines},plainpages=false,pdfpagelabels=true,
colorlinks=true,linkcolor={black},citecolor={black},pagecolor={black},
urlcolor={black},
pdftitle={DCPT with fallback},
pdfsubject={Prefetcher},
pdfauthor={Leif Tore Rusten, Stian Fredrikstad, Vegar K\aa sli},
pdfkeywords={dcpt,prefetcher}}
\hyphenation{op-tical net-works semi-conduc-tor}


\begin{document}
\title{DCPT with fallback}
\author{Leif Tore Rusten,
        Stian Fredrikstad and
        Vegar K\aa sli}

\markboth{DCPT with fallback}%
{DCPT with fallback}

\IEEEcompsoctitleabstractindextext{%
\begin{abstract}
This paper will explain some tweaks to the DCPT algorithm and compare speedups
to the original DCPT
\end{abstract}

\begin{IEEEkeywords}
DCPT, sequential, prefetcher
\end{IEEEkeywords}}

\maketitle
\IEEEdisplaynotcompsoctitleabstractindextext
\IEEEpeerreviewmaketitle

\section{Introduction}
Prefetcher has been a hot topic the last years since the memory gap has become so big.

\section{Methodology}
The test setup consist of the M5 simulator running on NTNU's computer
cluster kongull. M5 is setup to simulate the Alpha 21264 microarchitecture
plus the prefetcher described in this paper. A subset of the SPEC CPU2000
benchmarks where used for measuring speedup. These SPEC benchmarks are assumed
to be representative for real world workloads, making the generated speedup
results relevant for real world applications. This test setup is given
as part of the project work in the course TDT4260 at NTNU. In addition,
shell scripts for brute force search for optimal prefetcher parameters
where created.

The M5 simulator supplied had been modified by the course responsible such
that the programming interface had been abstracted into a simplified framework
allowing for an minimal learning curve and fast prefetcher development. The
framework also contained python scripts to run a lighter version of the
benchmarks which run on kongull. For tuning the different parameters shell
scripts where created which iterates over the parameter space and runs the
heavy version of the benchmarks, i.e. the same as on kongull. The command line
for an M5 execution looks as follows:

\begin{verbatim}
M5_CPU2000=lib/cpu2000
./m5/build/ALPHA_SE/m5.opt
--remote-gdb-port=0 -re --outdir
output/${b} m5/configs/example/se.py
--detailed --checkpoint-dir=lib/cp
--checkpoint-restore=1000000000
--at-instruction --caches --l2cache
--standard-switch --membus-width=8
--warmup-insts=10000000
--max-inst=100000000 --l2size=1MB
--membus-clock=400MHz
--mem-latency=30ns --bench=${b}
--prefetcher=on_access=true:policy=proxy
\end{verbatim}

In the command line above, \${b} is one of the benchmarks in the selected
subset of CPU2000. This subset was given by the supplied framework and consists
of the following benchmarks: ammp, art110, art470, wupwise, swim, applu,
galgel, apsi, bzip2\_source, bzip2\_graphic, bzip2\_program, twolf.

The statistical output from each benchmark is parsed to give the corresponding
IPC before using this to calculate speedup by comparing to the result
of no prefetcher. The collective speedup is presented as a harmonic mean of
the individual results. This is done by the framework, but a similar
functionality was developed in shell scripts for use as previously mentioned
and for custom cross-compiled programs. The shell scripts generates
speedups relative to a given entry, but if this entry is the case of no
prefetcher then the results are the same as for the framework.

The chosen benchmarks are a subset of CPU2000 with the characteristic that
they seem to be hard to get good speedup from ``for free'', making
them especially worthwhile for academic exploration since these workloads
seemingly constitues an opportunity for large performance gains.

\section{Delta Correlating Prediction Table (DCPT) with sequential fallback}
The algorithm for DCPT (Delta Correlating Prediction Table) is based on the pseudo code from \cite{dcptpaper}.
The algorithm is related to RPT, but the difference is that it looks for a pattern instead of one stride.
When the prefetcher gets an access, it saves the PC, and the stride in a table connected to each PC. When it has more than two elements in the list with strides, it starts looking for patterns.
If it finds a pattern in the list, it saves all the adresses that should be prefetched in another list with candidates for prefetch.
Then it runs through the list and checks if the adresses is not in the cache, in the mshr or that the adress is out of memory. If all the requirements are met, it issues a prefetch for it.

If it has less than two deltas in the list or if it cant find a pattern, it starts prefetching the next block like a sequential prefetcher.

Attempts to tweak this implementation:
\begin{itemize}
\item Make a copy of the queue to keep track of which adresses that still aren't fetched, to avoid "double fetching".
\item Make a lru which prevents the prefetcher from throwing out recently used history from the list.
\item Adjustments to the size of the list, and size of the delta list
\end{itemize}

\subsection{Pseudocode}

\begin{algorithm}[H]
\dontprintsemicolon
\KwIn{candidates}

\Begin{
 \ForEach{$I \in candidates$}{
  \If{$canPrefetch(I)$}{
   $issuePrefetch(I)$\; 
  }
 }
}
\caption{prefetch\label{pre}}
\end{algorithm}

\begin{algorithm}[H]
\dontprintsemicolon
\KwIn{stat}

\Begin{
 $i \leftarrow TABLE[stat.pc]$\;
 \If{$i \neq 0$}{
  $DELTA \leftarrow stat.mem\_addr - i.last\_mem\_addr$\;
  $i.deltas[ ] \leftarrow DELTA$\;
 }
 $i.last\_mem\_addr \leftarrow stat.mem\_addr$\;
 $TABLE[stat.pc] \leftarrow i$\;
 $candidates \leftarrow []$\;
 \If{$size(I.deltas) > 1$}{
   $candidates \leftarrow DCPT(i)$\;
   $prefetch(candidates)$\;
 }
 \If{$size(candidates) == 0$ {\bf and} $canPrefetch(NEXT\_BLOCK)$}{
  $issue\_prefetch(NEXT\_BLOCK)$\;
 }
}
\caption{prefetch\_access\label{pa}}
\end{algorithm}

\begin{algorithm}[H]
\dontprintsemicolon
\KwIn{addr}

\Begin{
 ${\bf return}$ $addr < MAX\_PHYS\_ADDR$ \\
	{\bf and} $!in\_mshr\_queue(addr)$ \\
	{\bf and} $!in\_cache(addr)$ \\
	{\bf and} $!in\_cache\_queue(addr)$\;
}
\caption{canPrefetch\label{cp}}
\end{algorithm}

\begin{algorithm}[H]
\dontprintsemicolon
\KwIn{i}

\Begin{
 $last\_delta\_i \leftarrow i.deltas.last$\;
 \ForEach{$U, V \in reversed(i.deltas)$}{
  \If{$U == i.deltas[last\_delta\_i]$ {\bf and} $V == i.deltas[last\_delta\_i - 1]$}{
   $match\_i \leftarrow indexof(U)$\;	
   ${\bf break}$\;
  }
 }
 $candidates \leftarrow []$\;
 \While{$match\_i < size(i.deltas)$}{
  $candidates \leftarrow i.last\_mem\_addr$ + $i.deltas[++match\_i]$\;
 }
 ${\bf return}$ $candidates$\;
}
\caption{DCPT\label{dcpt}}
\end{algorithm}



\section{Results}
\subsection{Delta Correlation Prediction Table}
The prefetcher was tested with and without the sequential fallback and as seen in \ref{fig:dcptspeed}. The harmonic mean for the prefetcher without the sequential fallback was 1.085, and when sequential fallback was applied the speedup increased to 1.101 which is significally better even though some of the programs lose speedup because of the sequential prefetching. These results was gotten when running the prefetcher with 120 entries in the table and 16 deltas for each entry. This was the optimal sizes that gives an overall best speedup.

\begin{figure}[H]
\begin{center}
\includegraphics[width=0.45\textwidth]{dcpt_speedup_plt}
\end{center}
\caption{\label{fig:dcptspeed} DCPT with and without sequential}
\end{figure}

It was proven that to keep a local queue to stop the prefetcher from prefetching a block two times had no impact to the speedup since the framework does this filtering already, but if the prefetcher should be used in a real world application this would be required.

\subsection{Sequential fallback}
In an effort to measure the overall impact of the sequential fallback
mechanism, the relative speedup for different prefetch degrees where
generated. These values where generated by shell scripts iteratively
compiling the prefetcher module with increasing degree before running
the heavy version benchmark (i.e. the same as on kongull) locally. The
result is presented in figure \ref{fig:fallback}.

\begin{figure}
\begin{center}
\includegraphics[width=0.45\textwidth]{sequential_failover_speedup_plt}
\end{center}
\caption{\label{fig:fallback} Relative speedup for sequential fallback}
\end{figure}

As seen from the figure, the speedup is normalized to the case for degree equal
to zero corresponding to a plain DCPT implementation with no fallback. The
small positiv effect from the fallback mechanism is seen to be rapidly
vanishing for increasing degrees. For degrees larger than 2 to be efficient
they would need to be good prefetches anyhow, creating an access pattern which
the DCPT scheme in any case would be able to detect and since start
prefetching. The fallback mechanism is therefore assumed mostly a fallback for
the cases where sequential parts of data are being read
(an assumed common case) but the DCPT scheme is still recuperating from a
previous pattern.

Also since only one block is prefetched, the risk of evicting a cache line
that would otherwise be used in the near future is marginalized. Still the
relative speedup from regular DCPT is not impressively large at 1.007, but
seeing as how sequential prefetching is trivial to implement it is a
well-worth investment for performance gain.

\section{Discussion}
\subsection{Using LRU to preserve history}
Since there are a 8kB memory limit it was looked into modifying the table in DCPT to use LRU. This was implemented by always storing the Tick that was given with the argument to prefetch\_access and use this to determine which pc that should be replaced by a new. Unfortunately this reduced the total speedup, this is possibly because there are more often hits in the table and that could give more hits in the deltas table, and this leads in some cases to more misses and less sequential prefetching, specifically the twolf which uses some have use of sequential prefetch had reduced speedup.

\subsection{Too aggressive prefetching}
The prefetcher implemented here is very aggressive and gives a lot of identified block, but only issues some percent of these. The reason why a block is identified but not issued is in some cases that the block is already in the queue, because of that a copy of the cache queue was implemented and checked before every prefetch, this helped a little, but there are still too many identified compared to issued. Some of the problem can also be that the prefetcher matches patterns with a large number of deltas between, this leads to that the prefetches identifies too many blocks every time prefetch\_access is called and since the queue is limited, some of the blocks that are actually needed are pushed out of the queue. %\TODO add discussion with results from pfjudge whats up? with real limit
\section{Conclusion}

\section{Further Work}
This section is optional, but might become large.

\subsection{Stuff we will look into before delivery (i.e. our plan forward)}
Continuing to tweak and implement the modified regular DCPT. The plan is
to work with this in March and cut off the gathering of results 1. April
and start focusing on writing the report and presentation.

Look into the generalization of DCPT as a tree-structure. If this gives
ok speedup then we will change the focus of the report into comparing this
approach with our version of DCPT. If performance is low, only mention the
design and do not analyse it further, instead focus on analysing our
DCPT version.

The tree-generalization of DCPT is outlined somewhat as follows. Each new PC
causing an access gets a tree-root-node and two tree-pointers allocated.
As this access is the first access for the PC, no history is available.
Therefore add a child node to the PC's root and associate the stride 1 with it.
Move one of the tree-pointers, cur\_prefetch to this child, and leave the other
pointer, true\_access, alone. At next access for this PC verify if this access
has been correctly prefetched by traversing from cur\_prefetch and up to the
root, checking if any node correspond to the access. If this is the case,
move true\_access to that node. If no such node is found, search the children
of true\_access and check for a stride giving this access. If child found, move
true\_access to this and start prefetching from this nodes subtree and move
cur\_prefetch accordingly. If no child found create a new child node and set
its stride to 1 as in the starting case. These starting or learning cases
are such that if the stride 1 is guessed wrong, correct the stride in the node
instead of creating a new child node, but after this set the node as no longer
learning.

\subsection{Stuff we won't have time to, but that might be good ideas}

%\appendices
%\section{}
%Appendix one text goes here.

% you can choose not to have a title for an appendix
% if you want by leaving the argument blank
%\section{}
%Appendix two text goes here.


% use section* for acknowledgement
\ifCLASSOPTIONcompsoc
  % The Computer Society usually uses the plural form
%  \section*{Acknowledgments}
\else
  % regular IEEE prefers the singular form
%  \section*{Acknowledgment}
\fi


% Can use something like this to put references on a page
% by themselves when using endfloat and the captionsoff option.
\ifCLASSOPTIONcaptionsoff
  \newpage
\fi



\begin{thebibliography}{1}

\bibitem{dcptpaper}
M.~Grann\ae s, M.~Jahre and L.~Natvig, \emph{Storage Efficient Hardware Prefetching using Delta Correlating Prediction Tables}, 2009.

\end{thebibliography}



% that's all folks
\end{document}


