\chapter{Further Work}
This section is optional, but might become large.

\section{Stuff we will look into before delivery (i.e. our plan forward)}
Continuing to tweak and implement the modified regular DCPT. The plan is
to work with this in March and cut off the gathering of results 1. April
and start focusing on writing the report and presentation.

Look into the generalization of DCPT as a tree-structure. If this gives
ok speedup then we will change the focus of the report into comparing this
approach with our version of DCPT. If performance is low, only mention the
design and do not analyse it further, instead focus on analysing our
DCPT version.

The tree-generalization of DCPT is outlined somewhat as follows. Each new PC
causing an access gets a tree-root-node and two tree-pointers allocated.
As this access is the first access for the PC, no history is available.
Therefore add a child node to the PC's root and associate the stride 1 with it.
Move one of the tree-pointers, cur\_prefetch to this child, and leave the other
pointer, true\_access, alone. At next access for this PC verify if this access
has been correctly prefetched by traversing from cur\_prefetch and up to the
root, checking if any node correspond to the access. If this is the case,
move true\_access to that node. If no such node is found, search the children
of true\_access and check for a stride giving this access. If child found, move
true\_access to this and start prefetching from this nodes subtree and move
cur\_prefetch accordingly. If no child found create a new child node and set
its stride to 1 as in the starting case. These starting or learning cases
are such that if the stride 1 is guessed wrong, correct the stride in the node
instead of creating a new child node, but after this set the node as no longer
learning.

\section{Stuff we won't have time to, but that might be good ideas}
