\documentclass[12pt,journal,compsoc]{IEEEtran}
\providecommand{\PSforPDF}[1]{#1}
\newcommand\MYhyperrefoptions{bookmarks=true,bookmarksnumbered=true,
pdfpagemode={UseOutlines},plainpages=false,pdfpagelabels=true,
colorlinks=true,linkcolor={black},citecolor={black},pagecolor={black},
urlcolor={black},
pdftitle={DCPT tweaks},
pdfsubject={Prefetcher},
pdfauthor={Leif Tore Rusten, Stian Fredrikstad, Vegar K\aa sli},
pdfkeywords={dcpt,rpt,prefetcher}}
\hyphenation{op-tical net-works semi-conduc-tor}


\begin{document}
\title{DCPT tweaks \\ Working title}
\author{Leif Tore Rusten,
        Stian Fredrikstad,
        and Vegar K\aa sli}

\markboth{DCPT tweaks}%
{DCPT tweaks}

\IEEEcompsoctitleabstractindextext{%
\begin{abstract}
%\boldmath
This paper will explain some tweaks to the DCPT algorithm and compare speedups to the "original" DCPT
\end{abstract}
% IEEEtran.cls defaults to using nonbold math in the Abstract.
% This preserves the distinction between vectors and scalars. However,
% if the journal you are submitting to favors bold math in the abstract,
% then you can use LaTeX's standard command \boldmath at the very start
% of the abstract to achieve this. Many IEEE journals frown on math
% in the abstract anyway. In particular, the Computer Society does
% not want either math or citations to appear in the abstract.

% Note that keywords are not normally used for peerreview papers.
\begin{IEEEkeywords}
DCPT, RPT, prefetcher
\end{IEEEkeywords}}


% make the title area
\maketitle


% To allow for easy dual compilation without having to reenter the
% abstract/keywords data, the \IEEEcompsoctitleabstractindextext text will
% not be used in maketitle, but will appear (i.e., to be "transported")
% here as \IEEEdisplaynotcompsoctitleabstractindextext when compsoc mode
% is not selected <OR> if conference mode is selected - because compsoc
% conference papers position the abstract like regular (non-compsoc)
% papers do!
\IEEEdisplaynotcompsoctitleabstractindextext
% \IEEEdisplaynotcompsoctitleabstractindextext has no effect when using
% compsoc under a non-conference mode.


% For peer review papers, you can put extra information on the cover
% page as needed:
% \ifCLASSOPTIONpeerreview
% \begin{center} \bfseries EDICS Category: 3-BBND \end{center}
% \fi
%
% For peerreview papers, this IEEEtran command inserts a page break and
% creates the second title. It will be ignored for other modes.
\IEEEpeerreviewmaketitle



\section{Introduction}

\section{"The Scheme"}
\subsection{Delta Correlating Prediction Table}
The algorithm for DCPT (Delta Correlating Prediction Table) is based on the pseudo code from \cite{dcptpaper}.
The implementation is very near RPT, but the difference is that it looks for a pattern instead of one stride.
When the it gets a request, it saves the (PC) program counter, and the stride in a table connected to each PC. When it has more than two elements in the list with strides, it starts looking for patterns.
If it finds a pattern in the list, it saves all the adresses that possibly should be prefetched in another list with candidates for prefetch.
Then it runs through the list and checks if the adresses is not in the cache, in the mshr or that the adress is out of memory. If all the requirements are met, it issues a prefetch for it.

Attempts to tweak this implementation:
- Make a copy of the queue to keep track of which adresses that still aren't fetched, to avoid "double fetching".
- Make a lru which prevents the prefetcher from throwing out recently used history from the list.
- Adjustments to the size of the list, and size of the delta list

\section{Results}
\section{Delta Correlation Prediction Table}
This implementation get a 8.6\% speedup with a table of 100 entries and 10 delta per entry.

The attempts to tweak this with other values have not been successful, but it gives around 8\% speedup.

Attempts to add a check to issue less prefetches does also lead to less speedup.

\section{Conclusion}
\appendices
\section{}
%Appendix one text goes here.

% you can choose not to have a title for an appendix
% if you want by leaving the argument blank
%\section{}
%Appendix two text goes here.


% use section* for acknowledgement
\ifCLASSOPTIONcompsoc
  % The Computer Society usually uses the plural form
%  \section*{Acknowledgments}
\else
  % regular IEEE prefers the singular form
%  \section*{Acknowledgment}
\fi


% Can use something like this to put references on a page
% by themselves when using endfloat and the captionsoff option.
\ifCLASSOPTIONcaptionsoff
  \newpage
\fi



\begin{thebibliography}{1}

\bibitem{dcptpaper}
M.~Grann\ae s, M.~Jahre and L.~Natvig, \emph{Storage Efficient Hardware Prefetching using Delta Correlating Prediction Tables}, 2009.

\end{thebibliography}

% biography section
% 
% If you have an EPS/PDF photo (graphicx package needed) extra braces are
% needed around the contents of the optional argument to biography to prevent
% the LaTeX parser from getting confused when it sees the complicated
% \includegraphics command within an optional argument. (You could create
% your own custom macro containing the \includegraphics command to make things
% simpler here.)
%\begin{biography}[{\includegraphics[width=1in,height=1.25in,clip,keepaspectratio]{mshell}}]{Michael Shell}
% or if you just want to reserve a space for a photo:

\begin{IEEEbiographynophoto}{Leif Tore Rusten}
Bastard
\end{IEEEbiographynophoto}

% if you will not have a photo at all:
\begin{IEEEbiographynophoto}{Stian Fredrikstad}
Retard
\end{IEEEbiographynophoto}

% insert where needed to balance the two columns on the last page with
% biographies
%\newpage

\begin{IEEEbiographynophoto}{Vegar Kaasli}
Wasted
\end{IEEEbiographynophoto}

% You can push biographies down or up by placing
% a \vfill before or after them. The appropriate
% use of \vfill depends on what kind of text is
% on the last page and whether or not the columns
% are being equalized.

%\vfill

% Can be used to pull up biographies so that the bottom of the last one
% is flush with the other column.
%\enlargethispage{-5in}



% that's all folks
\end{document}


