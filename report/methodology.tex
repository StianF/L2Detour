\chapter{Methodology}
%This section should describe how we attempt to solve our problems.

The methodology used in the beginning was to start with a sequential
prefetcher and briefly progress through prior art, i.e. an RPT implementation
and finally a DCPT implementation. During development the performance was
measured by running a M5-simulator testbench framework and the obtained test
results where evaluated against previous versions' results for
empirically determining the usefulness of the modification.

The prefetcher development has been driven by ideas from the group and it
has been a low barrier for implementing ideas. The ideas proposed have been
thought to be improvements mostly because of personal or group reflection over
the current prefetching mechanism and the memory hierarchy. Testing of
many different modifications and following performance evaluations have been
done rapidly.

So far the development can be seen as searching somewhat blindly for a good
solution. The proposed modifications have often been based on an insight
about the currently implemented prefetcher's (seq, RPT or DCPT) functionality.
Other modifications are mere tweaks of parameters for the different prefetcher
strategies for fine-tuning the performance metrics. Since the main focus of
our research is on the DCPT prefetcher and the development methodology
outlined above (basicly the group using its creativity) will probably only
give a noteworthy performance increase by chance. And even though such
modifications could have been based on a justified/rational idea spurred by
creativity, it would have been better if they where based on measurements and
novel memory access patterns derived from an analytical method.

This means that a shift of methodology probably is required to make truly
interesting discoveries academically and make it more likely to find and
exploit prefetching opportunities. For the remaining work the use of valgrind
and the inspection of which code snippets give cache misses and trying to
analyse these to find potential underlying factors/mechanisms for these
will become a more important tool as we run out of good a priori
insights/ideas.
